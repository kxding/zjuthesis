% \cleardoublepage

% {
%     \sectionnonum{附录}
%     \appendixsubsecmajornumbering
%     \subsection{实验参数设定}
%     本文提出的首帧替换策略、参考帧拼接策略以及ID注入方法均在8张 NVIDIA A800 显卡上进行训练。其中,首帧替换策略训练了约 20,000 个迭代步骤,参考帧拼接策略则训练约 450,000 步以达到最佳效果。所有实验均基于 DiT 模型进行,训练数据采自开源数据集 Koala36M,并从中随机抽取 100 万条样本用于模型训练。实验中所采用的关键参数配置如下所示:

% \begin{table}[h]
% \centering
% \caption{视频生成过程中的关键参数设定}
% \begin{tabular}{ll}
% \toprule
% \textbf{参数名称} & \textbf{取值} \\
% \midrule
% 图像宽度(width) & 672 \\
% 图像高度(height) & 384 \\
% 帧率(fps) & 15 \\
% 视频帧数(num\_frames) & 77 \\
% 引导尺度(guidance\_scale) & 5 \\
% 随机种子(seed) & 1234 \\
% 去噪步骤数(num\_inference\_steps) & 30 \\
% \bottomrule
% \end{tabular}
% \label{tab:video-gen-params}
% \end{table}
%     高分辨率参考图的视频生成方法则在16张A800显卡上进行训练,总训练迭代步数约为16,000步。该部分使用的数据集由10万对低质量与高质量视频片段构成,视频对通过文献\cite{wang2021real}中所提出的方法构建而成。

%     % End of appendix
%     \removeappendixsubsecmajornumbering
% }





